\documentclass{article}
\usepackage[utf8]{inputenc}
\usepackage{amsfonts}
\usepackage{amsmath}
\usepackage{amssymb}
\usepackage{amsthm}
\usepackage{esint}
\usepackage{ fancyhdr }
\usepackage{ enumitem }
\usepackage{amsmath}
\usepackage{amsthm}
\usepackage{amssymb}
\usepackage[linesnumbered , ruled , vlined]{algorithm2e}
\usepackage{ listings }
\usepackage{ xcolor }
\usepackage{ floatrow }
\usepackage{ graphicx }
\usepackage{fancyhdr}
\usepackage{listings}
\usepackage{caption}

\title{CS215 Assignment1 Task2}
\author{200050109-200050154}
\date{August 2021}

\begin{document}

\maketitle

\section{Solution}

\subsection{Part 1}
We used a loop to obtain $N:=10^6$ samples of $Z$ using a loop and making sure that the values lie in $\{0,1,\dots 25\}$. These values were stored as a frequency distribution indexed by the value of $Z$.
\begin{center}
\ffigbox [ 0.5 \textwidth ]{ \caption{Code 1} } { \includegraphics [ width =0.5\textwidth ] {code1.png}}  
\end{center}


\ffigbox [ 0.5 \textwidth ]{ \caption{values of Z} }{ \includegraphics [ width =1\textwidth,left ] {values11.png}} 
\ffigbox [ 0.5 \textwidth ]{ \caption{values of Z } } { \includegraphics [ width =1\textwidth,left ] {values12.png}}  

\subsection{Part 2}
Define $\lambda = \lambda_x + \lambda_y$. Analytically, the PMF of $Z$ is given by
\begin{equation*}
    P(Z) = \frac{ \lambda^k e^{-\lambda} } {k!}
\end{equation*}

\subsection{Part 3}
The values of $P(Z=k)$ and $P(\Hat{Z}=k)$ are very similar due to large value of $N$, and the graphs almost coincide.
\begin{center}
\ffigbox [ 0.5 \textwidth ]{ \caption{Graph 1} } { \includegraphics [ width =0.5\textwidth ] {empericalvstheoretical.png}}  
\end{center}

\subsection{Part 4}
We took the value $n$ obtained from the Poisson distribution with $\lambda=4$. Then this value was used to obtain the value $k$, using binomial distribution.

\begin{center}
\ffigbox [ 0.5 \textwidth ]{ \caption{Code 2} } { \includegraphics [ width =0.5\textwidth ] {code2.png}}  
\end{center}

\ffigbox [ 0.5 \textwidth ]{ \caption{values of Z} }{ \includegraphics [ width =1\textwidth,left ] {values21.png}} 
\ffigbox [ 0.5 \textwidth ]{ \caption{values of Z } } { \includegraphics [ width =1\textwidth,left ] {values22.png}}

\subsection{Part 5}
Theoretically, the value of the PMF obtained would be
\begin{equation*}
     P(Z) = \frac{(p\lambda)^k e^{-p\lambda} } {k!}
\end{equation*}
where $p:=0.8$ and $\lambda:=4$

\subsection{Part 5}
The values of $P(Z=k)$ and $P(\Hat{Z}=k)$ are very similar due to large value of $N$, and the graphs almost coincide.
\begin{center}
\ffigbox [ 0.5 \textwidth ]{ \caption{Graph 2} } { \includegraphics [ width =0.5\textwidth ] {thinning.png}}  
\end{center}

\end{document}
